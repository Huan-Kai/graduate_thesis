現今許多先進的測量方法已被開發來觀測組蛋白修飾的確切位置,但是這些方法針對每種細胞株上測量不同的組蛋白修飾是件非常耗時且昂貴的實驗,為了解決這個困境,在近期的研究中,已有許多方法透過基因序列來預測各種組蛋白修飾,然而基因序列是無法提供細胞株特異性的資訊,導致在預測不同細胞株的組蛋白修飾時會遭遇很大的瓶頸,因此本研究引入了基因序列以及甲基化數據,並且利用卷積神經網路來改善預測不同細胞株的組蛋白修飾。

本研究將"啟動模塊"的特殊架構加入卷積神經網路中,以此來同時抓取不同序列長度的特徵。接著,我們利用"分層批次"的學習演算法,有效地在不平衡的資料集中訓練模型。最後,我們將混和的基因序列及甲基化數據基於上述的方法進行訓練,並從中萃取出具有細胞株特異性資訊的特徵向量來預測組蛋白修飾。

在實驗中,結果顯示我們設計的模型超越了基線模型,尤其是在最不平衡的類別上得到最明顯的改善。此外,我們也對不同輸入數據對效能的影響進行分析,結果表明結合基因序列及甲基化數據進行訓練的模型,相較於單獨只使用一種數據訓練的模型,不管在哪一種組蛋白修飾上都有顯著的提升。最後,我們進一步利用視覺化工具發現我們設計的模型的確能學習到細胞株特異性的資訊。

綜合以上的結果及分析,我們證實了利用改良的卷積神經網路並且結合基因序列及甲基化數據是有助於預測不同細胞株的組蛋白變異。
% 我們分別利用基因序列及甲基化的數據進行訓練及特徵萃取,並且進一步的混合這兩者的數據進行預測

% 人類基因組錯綜複雜的調控一直是遺傳學上關注的議題之一,許多生物學家利用調控區域上的變化進行觀察及分析,來破譯各種不同的基因調控機制及疾病。組蛋白就是調控基因的成員之一,它能透過自身的修飾來改變染色質的特性,在表觀遺傳上啟到致關重要的作用,因此有許多先進的測量方法已被開發來觀測組蛋白修飾的確切位置,然而這些方法要針對不同的細胞株測量各種組蛋白修飾是件非常耗時且昂貴的實驗,為了解決這個困境,現有的方法利用大量公開的資料集對未測量或是丟失的資料進行預測,以幫助後續更完整的分析。

% 因此為了加強對不同細胞株的預測,本研究加入了具有細胞株甲基化數據並且結合基因序列來預測組蛋白修飾確切的結合位點。

% 這是因為人類所有細胞株中的基因序列都是相同的,以至於無法提供任何有關細胞特異性的資訊,

% 在發表過的研究中,已有許多方法成功地利用基因序列來預測各種組蛋白修飾,並且得到不錯的成績,然而單純只使用基因序列來預測不同細胞株的組蛋白修飾會遭遇到很大的瓶頸,因此本研究引入了具有細胞株特異性的甲基化及基因序列的數據,並且透過新穎的卷積神經網路架構來改善預測組蛋白修飾的結合位點。

% 但是同時要對不同細胞株的染色質預測時,單純只使用基因序列的資料,這是因為人類所有細胞的基因序列都是相同的,然而染色質會因為細胞株的特異性有所不同。

% 因此本研究提出了基於深度學習的方法從基因序列及甲基化的資料中萃取特徵,來預測組蛋白修飾確切的結合位點

% DNA + DL --> methylation --> 引出我們到底怎麼做的 --> 模型的改善 --> 結果及結論

\begin{flushleft}
\mbox{{\bf 關鍵字}:卷積神經網路、資料差補、組蛋白修飾、甲基化、細胞株特異性}
\end{flushleft}
