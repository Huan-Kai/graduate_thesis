%% // nckuee.sty 定義 // cbj

% 產生論文封面
\nckuEEtitlepage
% 產生口試委員會簽名單
%\nckuEEoralpage
% 產生口試委員簽名單(en)
% \nckuEEenoralpage

%\newpage
%\setcounter{page}{1}
%\pagenumbering{roman}

%%%%%%%%%%%%%%%%%%%%%%%%%%%%%%%
%       封面內頁
%%%%%%%%%%%%%%%%%%%%%%%%%%%%%%%
% % unmark to add inner cover
%\newpage
%\thispagestyle{empty}
%\thispagestyle{EmptyWaterMarkPage}
%\nckuEEtitlepage


%%%%%%%%%%%%%%%%%%%%%%%%%%%%%%%
%       中文摘要
%%%%%%%%%%%%%%%%%%%%%%%%%%%%%%%

% 可以利用如下自定義的command (定義在nckuee.sty)
% ======
%\begin{zhAbstract}  %中文摘要
%現今許多先進的測量方法已被開發來觀測組蛋白修飾的確切位置,但是這些方法針對每種細胞株上測量不同的組蛋白修飾是件非常耗時且昂貴的實驗,為了解決這個困境,在近期的研究中,已有許多方法透過基因序列來預測各種組蛋白修飾,然而基因序列是無法提供細胞株特異性的資訊,導致在預測不同細胞株的組蛋白修飾時會遭遇很大的瓶頸,因此本研究引入了基因序列以及甲基化數據,並且利用卷積神經網路來改善預測不同細胞株的組蛋白修飾。

本研究將"啟動模塊"的特殊架構加入卷積神經網路中,以此來同時抓取不同序列長度的特徵。接著,我們利用"分層批次"的學習演算法,有效地在不平衡的資料集中訓練模型。最後,我們將混和的基因序列及甲基化數據基於上述的方法進行訓練,並從中萃取出具有細胞株特異性資訊的特徵向量來預測組蛋白修飾。

在實驗中,結果顯示我們設計的模型超越了基線模型,尤其是在最不平衡的類別上得到最明顯的改善。此外,我們也對不同輸入數據對效能的影響進行分析,結果表明結合基因序列及甲基化數據進行訓練的模型,相較於單獨只使用一種數據訓練的模型,不管在哪一種組蛋白修飾上都有顯著的提升。最後,我們進一步利用視覺化工具發現我們設計的模型的確能學習到細胞株特異性的資訊。

綜合以上的結果及分析,我們證實了利用改良的卷積神經網路並且結合基因序列及甲基化數據是有助於預測不同細胞株的組蛋白變異。
% 我們分別利用基因序列及甲基化的數據進行訓練及特徵萃取,並且進一步的混合這兩者的數據進行預測

% 人類基因組錯綜複雜的調控一直是遺傳學上關注的議題之一,許多生物學家利用調控區域上的變化進行觀察及分析,來破譯各種不同的基因調控機制及疾病。組蛋白就是調控基因的成員之一,它能透過自身的修飾來改變染色質的特性,在表觀遺傳上啟到致關重要的作用,因此有許多先進的測量方法已被開發來觀測組蛋白修飾的確切位置,然而這些方法要針對不同的細胞株測量各種組蛋白修飾是件非常耗時且昂貴的實驗,為了解決這個困境,現有的方法利用大量公開的資料集對未測量或是丟失的資料進行預測,以幫助後續更完整的分析。

% 因此為了加強對不同細胞株的預測,本研究加入了具有細胞株甲基化數據並且結合基因序列來預測組蛋白修飾確切的結合位點。

% 這是因為人類所有細胞株中的基因序列都是相同的,以至於無法提供任何有關細胞特異性的資訊,

% 在發表過的研究中,已有許多方法成功地利用基因序列來預測各種組蛋白修飾,並且得到不錯的成績,然而單純只使用基因序列來預測不同細胞株的組蛋白修飾會遭遇到很大的瓶頸,因此本研究引入了具有細胞株特異性的甲基化及基因序列的數據,並且透過新穎的卷積神經網路架構來改善預測組蛋白修飾的結合位點。

% 但是同時要對不同細胞株的染色質預測時,單純只使用基因序列的資料,這是因為人類所有細胞的基因序列都是相同的,然而染色質會因為細胞株的特異性有所不同。

% 因此本研究提出了基於深度學習的方法從基因序列及甲基化的資料中萃取特徵,來預測組蛋白修飾確切的結合位點

% DNA + DL --> methylation --> 引出我們到底怎麼做的 --> 模型的改善 --> 結果及結論

\begin{flushleft}
\mbox{{\bf 關鍵字}:卷積神經網路、資料差補、組蛋白修飾、甲基化、細胞株特異性}
\end{flushleft}
 % // 可以引入front_cabstract.tex檔案或在此編輯 // cbj
%\end{zhAbstract}

% ...等
% ======

% 在此直接定義如下
%%%%%%%%%%%%%%%%
%
\newpage
% // HongJhe 頁碼起始
\setcounter{page}{1}
\pagenumbering{roman}
% create an entry in table of contents for 中文摘要
\phantomsection % for hyperref to register this
\addcontentsline{toc}{chapter}{\nameCabstract}
% aligned to the center of the page
\begin{center}
% font size (relative to 12 pt):
% \large (14pt) < \Large (18pt) < \LARGE (20pt) < \huge (24pt)< \Huge (24 pt)
% Set the line spacing to single for the names (to compress the lines)
\renewcommand{\baselinestretch}{1}   %行距 1 倍
% it needs a font size changing command to be effective
\LARGE{\zhTitle}\\  %中文題目
\vspace{0.83cm}
% \makebox is a text box with specified width;
% option s: stretch
% use \makebox to make sure
% each text field occupies the same width
%\makebox[1.5cm][c]{\large{學生:}}
\hspace{0.5in}
\renewcommand{\thefootnote}{\fnsymbol{footnote}}
\makebox[3.5cm][l]{\large{\authorZhName\footnote[1]{}}}\footnotetext[1]{{學生}} % 學生中文姓名
%\hfill
%
%\makebox[3cm][c]{\large{指導教授:}}
\makebox[3.5cm][l]{\large{\advisorZhName\footnote[2]{}}}\footnotetext[2]{{指導教授}} \\ %指導教授中文姓名
%
\vspace{0.42cm}
%
\large{\zhUniv}\large{\zhDepartmentName}\\ %校名系所名
\vspace{0.83cm}
%\vfill
\makebox[2.7cm][c]{\large{摘要}}
\end{center}
% Resume the line spacing to the desired setting
\renewcommand{\baselinestretch}{\mybaselinestretch}   %恢復原設定
%it needs a font size changing command to be effective
% restore the font size to normal
\normalsize
%%%%%%%%%%%%%
\par  % 摘要首段空格 by SianJhe
現今許多先進的測量方法已被開發來觀測組蛋白修飾的確切位置,但是這些方法針對每種細胞株上測量不同的組蛋白修飾是件非常耗時且昂貴的實驗,為了解決這個困境,在近期的研究中,已有許多方法透過基因序列來預測各種組蛋白修飾,然而基因序列是無法提供細胞株特異性的資訊,導致在預測不同細胞株的組蛋白修飾時會遭遇很大的瓶頸,因此本研究引入了基因序列以及甲基化數據,並且利用卷積神經網路來改善預測不同細胞株的組蛋白修飾。

本研究將"啟動模塊"的特殊架構加入卷積神經網路中,以此來同時抓取不同序列長度的特徵。接著,我們利用"分層批次"的學習演算法,有效地在不平衡的資料集中訓練模型。最後,我們將混和的基因序列及甲基化數據基於上述的方法進行訓練,並從中萃取出具有細胞株特異性資訊的特徵向量來預測組蛋白修飾。

在實驗中,結果顯示我們設計的模型超越了基線模型,尤其是在最不平衡的類別上得到最明顯的改善。此外,我們也對不同輸入數據對效能的影響進行分析,結果表明結合基因序列及甲基化數據進行訓練的模型,相較於單獨只使用一種數據訓練的模型,不管在哪一種組蛋白修飾上都有顯著的提升。最後,我們進一步利用視覺化工具發現我們設計的模型的確能學習到細胞株特異性的資訊。

綜合以上的結果及分析,我們證實了利用改良的卷積神經網路並且結合基因序列及甲基化數據是有助於預測不同細胞株的組蛋白變異。
% 我們分別利用基因序列及甲基化的數據進行訓練及特徵萃取,並且進一步的混合這兩者的數據進行預測

% 人類基因組錯綜複雜的調控一直是遺傳學上關注的議題之一,許多生物學家利用調控區域上的變化進行觀察及分析,來破譯各種不同的基因調控機制及疾病。組蛋白就是調控基因的成員之一,它能透過自身的修飾來改變染色質的特性,在表觀遺傳上啟到致關重要的作用,因此有許多先進的測量方法已被開發來觀測組蛋白修飾的確切位置,然而這些方法要針對不同的細胞株測量各種組蛋白修飾是件非常耗時且昂貴的實驗,為了解決這個困境,現有的方法利用大量公開的資料集對未測量或是丟失的資料進行預測,以幫助後續更完整的分析。

% 因此為了加強對不同細胞株的預測,本研究加入了具有細胞株甲基化數據並且結合基因序列來預測組蛋白修飾確切的結合位點。

% 這是因為人類所有細胞株中的基因序列都是相同的,以至於無法提供任何有關細胞特異性的資訊,

% 在發表過的研究中,已有許多方法成功地利用基因序列來預測各種組蛋白修飾,並且得到不錯的成績,然而單純只使用基因序列來預測不同細胞株的組蛋白修飾會遭遇到很大的瓶頸,因此本研究引入了具有細胞株特異性的甲基化及基因序列的數據,並且透過新穎的卷積神經網路架構來改善預測組蛋白修飾的結合位點。

% 但是同時要對不同細胞株的染色質預測時,單純只使用基因序列的資料,這是因為人類所有細胞的基因序列都是相同的,然而染色質會因為細胞株的特異性有所不同。

% 因此本研究提出了基於深度學習的方法從基因序列及甲基化的資料中萃取特徵,來預測組蛋白修飾確切的結合位點

% DNA + DL --> methylation --> 引出我們到底怎麼做的 --> 模型的改善 --> 結果及結論

\begin{flushleft}
\mbox{{\bf 關鍵字}:卷積神經網路、資料差補、組蛋白修飾、甲基化、細胞株特異性}
\end{flushleft}
 % // 可以引入front_eabstract.tex檔案或在此編輯 // cbj



%%%%%%%%%%%%%%%%%%%%%%%%%%%%%%%
%       英文摘要
%%%%%%%%%%%%%%%%%%%%%%%%%%%%%%%
%
%[method 1]

% 可以利用如下自定義的command (定義在nckuee.sty)
% ======
%\begin{enAbstract}  %英文摘要
%Nowadays, many advanced measurement methods have been developed to observe the exact binding of histone modifications. However, it is a very time-consuming and expensive experiment to measure different histone modifications on each cell line. In order to solve this dilemma, In recent researches, there have been many methods to predict various histone modifications through DNA sequences. However, DNA sequences cannot provide any cell line-specific information, which leads to a big bottleneck in predicting histone modifications in different cell lines. Therefore, this study introduced DNA sequences and methylation data, and utilized convolutional neural network to improve the prediction of histone modifications in different cell lines.

In this research, the special structure of the "inception module" is added to the convolutional neural network to capture features of different sequence lengths at the same time. Next, we use the "stratified mini-batch" mechanism to effectively train the model on the unbalanced dataset. We finally train the mixed DNA sequences and methylation data based on the above methods, and extract feature vectors with cell line-specific information to predict histone modifications.

In the experiment, the results showed that the model we designed outperforms the baseline model, especially in the most unbalanced class to get the most obvious improvement. Besides, we analyze the effects of different input data for performance. The experimental results show that the model trained with DNA sequence and methylation data has significantly better performance in any histone modification than amodel trained with only one type of data. Finally, we further used visualization tools to find out that the model we designed could indeed learn cell line-specific information.

Based on the above results and analysis, we confirmed that the use of an improved convolutional neural network combined with DNA sequence and methylation data is helpful for predicting histone variation in different cell lines.

\begin{flushleft}
{{\bf Keywords}: Convolutional Neural Network, Data Imputation, Histone Modification, DNA methylation, Specificity of Cell Line}
\end{flushleft}
 % // 可以引入front_eabstract.tex檔案或在此編輯 // cbj
%\end{enAbstract}

%[method 2]
\newpage
% create an entry in table of contents for 英文摘要
\phantomsection % for hyperref to register this
\addcontentsline{toc}{chapter}{\nameEabstract} % // HongJhe marked

% aligned to the center of the page
\begin{center}
% font size:
% \large (14pt) < \Large (18pt) < \LARGE (20pt) < \huge (24pt)< \Huge (24 pt)
% Set the line spacing to single for the names (to compress the lines)
\renewcommand{\baselinestretch}{1}   %行距 1 倍
%\large % it needs a font size changing command to be effective
\LARGE{\enTitle}\\  %英文題目
\vspace{0.83cm}
% \makebox is a text box with specified width;
% option s: stretch
% use \makebox to make sure
% each text field occupies the same width
%\makebox[2cm][s]{\large{Student: }}
\hspace{0.45in}
\renewcommand{\thefootnote}{\fnsymbol{footnote}}
\makebox[5cm][l]{\large{\authorEnName\footnote[1]{}}}\footnotetext[1]{{Student}} % 學生英文姓名
%\hfill
%
%\makebox[2cm][s]{\large{Advisor: }}
\makebox[5cm][l]{\large{\advisorEnName\footnote[2]{}}}\footnotetext[2]{{Advisor}} \\ %教授英文姓名
%
\vspace{0.42cm}
\large{\enDepartmentName}\\ %英文系所全名
%
\large{\enUniv}\\  %英文校名
\vspace{0.83cm}
%\vfill
%
\large{\nameEabstractc}\\
%\vspace{0.5cm}
\end{center}

% Resume the line spacing the desired setting
\renewcommand{\baselinestretch}{\mybaselinestretch}   %恢復原設定
%\large %it needs a font size changing command to be effective
% restore the font size to normal
\normalsize
%%%%%%%%%%%%%
Nowadays, many advanced measurement methods have been developed to observe the exact binding of histone modifications. However, it is a very time-consuming and expensive experiment to measure different histone modifications on each cell line. In order to solve this dilemma, In recent researches, there have been many methods to predict various histone modifications through DNA sequences. However, DNA sequences cannot provide any cell line-specific information, which leads to a big bottleneck in predicting histone modifications in different cell lines. Therefore, this study introduced DNA sequences and methylation data, and utilized convolutional neural network to improve the prediction of histone modifications in different cell lines.

In this research, the special structure of the "inception module" is added to the convolutional neural network to capture features of different sequence lengths at the same time. Next, we use the "stratified mini-batch" mechanism to effectively train the model on the unbalanced dataset. We finally train the mixed DNA sequences and methylation data based on the above methods, and extract feature vectors with cell line-specific information to predict histone modifications.

In the experiment, the results showed that the model we designed outperforms the baseline model, especially in the most unbalanced class to get the most obvious improvement. Besides, we analyze the effects of different input data for performance. The experimental results show that the model trained with DNA sequence and methylation data has significantly better performance in any histone modification than amodel trained with only one type of data. Finally, we further used visualization tools to find out that the model we designed could indeed learn cell line-specific information.

Based on the above results and analysis, we confirmed that the use of an improved convolutional neural network combined with DNA sequence and methylation data is helpful for predicting histone variation in different cell lines.

\begin{flushleft}
{{\bf Keywords}: Convolutional Neural Network, Data Imputation, Histone Modification, DNA methylation, Specificity of Cell Line}
\end{flushleft}
 % // 可以引入front_eabstract.tex檔案或在此編輯 // cbj


%%%%%%%%%%%%%%%%%%%%%%%%%%%%%%%
%       誌謝
%%%%%%%%%%%%%%%%%%%%%%%%%%%%%%%
%
% Acknowledgment
\newpage
\phantomsection % for hyperref to register this
%\addcontentsline{toc}{chapter}{\nameAcknc}

\begin{zhAckn}  %誌謝
Add your acknowledgements here.

\begin{flushright}
\mbox{Syu-Min Cyu}
\end{flushright} % // 可以引入front_ackn.tex檔案或在此編輯 // cbj
\end{zhAckn}

%\chapter*{\nameAckn} %\makebox{} is fragile; need protect
%Add your acknowledgements here.

\begin{flushright}
\mbox{Syu-Min Cyu}
\end{flushright} % // 可以引入my_ackn.tex檔案或在此編輯 // cbj
%%testjsjtoejiojsoijtoijos

%%%%%%%%%%%%%%%%%%%%%%%%%%%%%%%
%       目錄
%%%%%%%%%%%%%%%%%%%%%%%%%%%%%%%
%
% Table of contents
\newpage
\renewcommand{\contentsname}{\nameToc}
%\makebox{} is fragile; need protect
\phantomsection % for hyperref to register this
\addcontentsline{toc}{chapter}{\nameTocc}
\tableofcontents

%%%%%%%%%%%%%%%%%%%%%%%%%%%%%%%
%       表目錄
%%%%%%%%%%%%%%%%%%%%%%%%%%%%%%%
%
% List of Tables
\newpage
\renewcommand{\listtablename}{\nameLot}
%\makebox{} is fragile; need protect
\phantomsection % for hyperref to register this
\addcontentsline{toc}{chapter}{\nameLotc}
\listoftables

%%%%%%%%%%%%%%%%%%%%%%%%%%%%%%%
%       圖目錄
%%%%%%%%%%%%%%%%%%%%%%%%%%%%%%%
%
% List of Figures
\newpage
\renewcommand{\listfigurename}{\nameTof}
%\makebox{} is fragile; need protect
\phantomsection % for hyperref to register this
\addcontentsline{toc}{chapter}{\nameTofc}
\listoffigures
%%%%%%%%%%%%%%%%%%%%%%%%%%%%%%%
%       符號說明
%%%%%%%%%%%%%%%%%%%%%%%%%%%%%%%
%
% Symbol list
% define new environment, based on standard description environment
% adapted from p.60~64, <<The LaTeX Companion>>, 1994, ISBN 0-201-54199-8

%\newcommand{\SymEntryLabel}[1]%
%  {\makebox[3cm][l]{#1}}
%%
%\newenvironment{SymEntry}
%   {\begin{list}{}%
%       {\renewcommand{\makelabel}{\SymEntryLabel}%
%        \setlength{\labelwidth}{3cm}%
%        \setlength{\leftmargin}{\labelwidth}%
%        }%
%   }%
%   {\end{list}}
%%%
%\newpage
%\chapter*{\nameSlist} %\makebox{} is fragile; need protect
%\phantomsection % for hyperref to register this
%\addcontentsline{toc}{chapter}{\nameSlistc}
%%
% this file is encoded in utf-8
% v2.0 (Apr. 5, 2009)
%  各符號以 \item[] 包住,然後接著寫說明
% 如果符號是數學符號,應以數學模式$$表示,以取得正確的字體
% 如果符號本身帶有方括號,則此符號可以用大括號 {} 包住保護
\begin{SymEntry}

\item[OLED]
Organic Light Emitting Diode

\item[$E$]
energy

\item[$e$]
the absolute value of the electron charge, $1.60\times10^{-19}\,\text{C}$
 
\item[$\mathscr{E}$]
electric field strength (V/cm)

\item[{$A[i,j]$}]
the  element of the matrix $A$ at $i$-th row, $j$-th column\\
矩陣 $A$ 的第 $i$ 列,第 $j$ 行的元素

\end{SymEntry}

\newpage
\setcounter{page}{1}
\pagenumbering{arabic}
