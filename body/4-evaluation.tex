\hspace{24pt}
\renewcommand{\baselinestretch}{1.5}

In this chapter, we will explain the detail of experiments. First, we describe the experimental environment in Section\ref{environment}. Second, we introduce the dataset from ENCODE, the method how to select within many datasets in Section\ref{dataset}. Third, before showing the performance, we early depict evaluation metrics in Section\ref{metric}. Fourth, we show experiment results from different models to compare their performance in Section\ref{performance}. Finally, we further analyze model by visualization of feature vectors in Section\ref{discuss}.

\section{Environment} \label{environment}
We use PyTorch, one of the most deep learning libraries \cite{paszke2017automatic}, to construct architecture of our model. And, the model designed by PyTorch can be trained and operated on a CUDA-capable Nvidia GPU to upgrade experimental efficiency. Our GPU is GEFORCE RTX 2070 SUPER. In order to more conveniently monitor model, we use Weights \& Biases (wandb) \cite{wandb} for logging hyperparameters and output metrics, then quickly visualize and compare results with dashboard on website.

\section{Dataset} \label{dataset}
We download datasets from the ENCODE project. This project is being conducted by ENCODE consortium, funded by the National Human Genome Research Institute. The goal of ENCODE is to build a comprehensive list of functional elements in the human genome. To high quality of analysis, ENCODE investigators employ a variety of assays and methods to identify various functional elements, and all generated data is available on ENOCDE website \cite{davis2018encyclopedia}.

We select three cell lines as datasets, including A549, GM12878 and K562. However, each epigenetic data on ENCODE website may be measured by different laboratories, while setting the different parameters in experiment that influence generated data. Hence, we select the dataset according to the warnings represented on ENCODE website. These warnings may indicate an error in the experimental metadata, or may indicate that the data itself does not meet some aspect of the consortium’s standards. There are three flag’s colors corresponding to the severity of the problem \cite{davis2018encyclopedia}.

\begin{figure}[H]
    \centering
    \includegraphics[width=0.5\columnwidth]{body/figure/figure13.png}
    \captionsetup{labelfont=bf}
    \renewcommand{\baselinestretch}{1.0}
    \caption[Example of warning levels]{This figure is screenshotted from ENCODE website. The example shows that different levels of severity in low read depth, which acts as metric to evaluate the quality of sequencing experiment.}
    \label{f13}
\end{figure}

Therefore, we choose datasets, that are less amount of warnings and lower level of warnings, to further analysis.

\section{Evaluation Metrics} \label{metric}
We use two metrics to evaluate our model. There are respectively area under the receiver operating characteristics curve (ROC) and area under the precision-recall curve (PRC). These metrics are common evaluations of a classifier’s prediction performance, which are threshold-free measures.

Before introducing area under ROC and area under PRC, we first introduce confusion matrix because these metrics are composed of the elements of this matrix. The confusion matrix consists of true positive (TP), false positive (FP), true negative (TN) and false negative (FN). shown in Figure\ref{f14}. Each row of the matrix represents the instances in an actual class while each column represents the instances in a predicted class, or vice versa. The several basic metrics are calculated by elements of confusion matrix, including precision, recall (also called sensitivity) and specificity \cite{saito2015precision}, shown in function(2) ,(3) and (4).

\begin{figure}[H]
    \centering
    \includegraphics[width=0.65\columnwidth]{body/figure/figure14.png}
    \captionsetup{labelfont=bf}
    % \renewcommand{\baselinestretch}{1.0}
    \caption[Confusion matrix]{Confusion matrix.}
    \label{f14}
\end{figure}

\vspace{-1cm}
\[Precision : \frac{TP}{TP+FP}\ (2)\]
\[Recall(sensitivity) : \frac{TP}{TP+FN}\ (3),\ Specificity : \frac{TN}{TN+FP}\ (4)\]

Then, ROC is plotted by sensitivity and specificity. PRC is plotted by precision and recall. Areas under these curves are metrics that evaluating our model. In addition, these curves provide useful information of baseline (random classifier). The baseline showed in ROC is a diagonal line, whereas the baseline showed in PRC is dynamic. The baseline of PRC is determined by the ratio of positives (P) and negatives (N) as y = P / (P+N) \cite{saito2015precision}, shown in Figure\ref{f15}.

\begin{figure}[H]
    \centering
    \includegraphics[width=1\columnwidth]{body/figure/figure15.png}
    \vspace{-1cm}
    \captionsetup{labelfont=bf}
    \renewcommand{\baselinestretch}{1.0}
    \caption[ROC and PRC]{Solid line represents monitored model. Dotted line represents baseline (random classifier). And, area under these curves represents performance of these models. (figure reproduced from \cite{rocprc})}
    \label{f15}
\end{figure}

\section{Prediction Performance} \label{performance}
In order to validate whether introducing techniques and features are helpful to our model. we design a series of following experiments to observe results. Finally, we will show the results of best model trained on all cell lines in Section\ref{cross}.

\subsection{Baseline}
We construct a CNN according to the architecture of DeepSEA \cite{zhou2015predicting}, which acts as baseline. The baseline is depicted in Figure\ref{f16}, and is trained on same training methodology with our model, described in Section\ref{method}. Only difference is that amount of neurons in the last fully connected layer. We focus on data imputation of six histone modifications, instead of interaction of various chromatin states. Therefore, we change the amount of neurons in the last fully connected layer to six, that the number is amount of histone modifications we want to predict.

\begin{figure}[H]
    \centering
    \includegraphics[width=0.6\columnwidth]{body/figure/figure16.png}
    \captionsetup{labelfont=bf}
    \renewcommand{\baselinestretch}{1.0}
    \caption[Framework of baseline]{The framework of baseline is same with DeepSEA.}
    \label{f16}
\end{figure}

\subsection{Inception Module}
To experimental efficiency, we first select nine partitions (1369713 windows) of K562 to train on baseline and our model, and then evaluate on one partition (152191 windows) of K562 to compare the results of these models. Accroding to Section\ref{metric}, we use area under ROC (AUROC) and area under PRC (AUPRC) as evaluation metrics.

Here, we want to observe the change of results after adding inception module into our model. Our model is named “Inception” in following tables. Apparently, we improve the prediction in all cases compared to baseline, shown in Table\ref{t4} and Table\ref{t5}.

\subsection{Stratified Mini-batch} \label{strat}
According to same settings with above experiment, we want to test whether stratified mini-batch is good for our task, that is imbalanced multi-label classification, shown in Table\ref{t4} and Table\ref{t5}. In the following tables, we use subscript to represent the model adding stratified mini-batch. In addition, we test on not only our model but also baseline, in order to check whether this mechanism only helps on special case.

\begin{table}[H]%加入table環境指令以控制表格的位置、編號與標題,[h]代表將表格置於here,其他位置的標示請參考手冊
    \centering
    \begin{tabular}{lcccc}
    \hline
    Model & Baseline & $Baseline_{stratified}$ & Inception & $Inception_{stratified}$ \\\hline
    H3K4me3 & 0.979 & 0.9819 & 0.9902 & \textbf{0.9927} \\
    H3K27ac & 0.9278 & 0.9334 & 0.9488 & \textbf{0.9549} \\
    H3K4me1 & 0.9672 & 0.9718 & 0.9745 & \textbf{0.9787} \\
    H3K36me3 & 0.985 & 0.9854 & 0.9857 & \textbf{0.9886} \\
    H3K9me3 & 0.9283 & 0.9429 & 0.9726 & \textbf{0.9762} \\
    H3K27me3 & 0.9923 & 0.9938 & 0.9964 & \textbf{0.9969} \\\hline
    \end{tabular}
    \captionsetup{labelfont=bf}
    \renewcommand{\baselinestretch}{1.0}
    \caption[Comparison of baseline and inception with AUROC]{This table represents performance with AUROC. Best predictive performance of each histone modification is shown in bold.}
    \label{t4}
\end{table}

\begin{table}[H]%加入table環境指令以控制表格的位置、編號與標題,[h]代表將表格置於here,其他位置的標示請參考手冊
    \centering
    \begin{tabular}{lcccc}
    \hline
    Model & Baseline & $Baseline_{stratified}$ & Inception & $Inception_{stratified}$ \\\hline
    H3K4me3 & 0.8995 & 0.9089 & 0.9365 & \textbf{0.9474} \\
    H3K27ac & 0.723 & 0.7385 & 0.7754 & \textbf{0.7934} \\
    H3K4me1 & 0.9388 & 0.9465 & 0.9501 & \textbf{0.9588} \\
    H3K36me3 & 0.8886 & 0.8905 & 0.9082 & \textbf{0.9208} \\
    H3K9me3 & 0.541 & 0.5911 & 0.7641 & \textbf{0.8114} \\
    H3K27me3 & 0.9905 & 0.9925 & 0.996 & \textbf{0.9965} \\\hline
    \end{tabular}
    \captionsetup{labelfont=bf}
    \renewcommand{\baselinestretch}{1.0}
    \caption[Comparison of baseline and inception with AUPRC]{This table represents performance with AUPRC. Best predictive performance of each histone modification is shown in bold.}
    \label{t5}
\end{table}

Regardless of the frameworks in above table, model adding this mechanism outperform original model. And, the model, which introduces inception module and stratified mini-batch into, gets the best results in above experiments. Hence, we decide to use this model to extract features from DNA sequences and DNA methylation signal, and then use these features to predict histone modifications.

\subsection{DNA methylation}
After confirming these techniques is helpful, we mainly want to observe how much different input features influence the results of model. We separately extract features from only DNA sequences, only DNA methylation signal and both, as well as represent them by subscript in following tables. It is same with above experiments that all models are trained on K562. In addition, we further check whether the framework of DeepSEA can also improve predition by adding features from DNA methylation signal, shown in Table\ref{t6}, Table\ref{t7}, Table\ref{t8} and Table\ref{t9}.

\begin{table}[H]%加入table環境指令以控制表格的位置、編號與標題,[h]代表將表格置於here,其他位置的標示請參考手冊
    \centering
    \begin{tabular}{lccc}
    \hline
    Model & $Baseline_{Meth}$ & $Baseline_{DNA}$ & $Baseline_{DNA+Meth}$ \\\hline
    H3K4me3 & 0.8963 & \underline{0.9741} & \textbf{0.9819} \\
    H3K27ac & 0.8475 & \underline{0.9172} & \textbf{0.9334} \\
    H3K4me1 & \underline{0.9435} & 0.8943 & \textbf{0.9718} \\
    H3K36me3 & \underline{0.9766} & 0.8982 & \textbf{0.9854} \\
    H3K9me3 & 0.7912 & \textbf{\underline{0.9468}} & 0.9429 \\
    H3K27me3 & \underline{0.9612} & 0.9287 & \textbf{0.9938} \\\hline
    \end{tabular}
    \captionsetup{labelfont=bf}
    \renewcommand{\baselinestretch}{1.0}
    \caption[Comparison of different inputs of baseline with AUROC]{This table represents performance with AUROC. Best predictive performance of each histone modification is shown in bold.}
    \label{t6}
\end{table}

\begin{table}[H]%加入table環境指令以控制表格的位置、編號與標題,[h]代表將表格置於here,其他位置的標示請參考手冊
    \centering
    \begin{tabular}{lccc}
    \hline
    Model & $Baseline_{Meth}$ & $Baseline_{DNA}$ & $Baseline_{DNA+Meth}$ \\\hline
    H3K4me3 & 0.6433 & \underline{0.8562} & \textbf{0.9089} \\
    H3K27ac & 0.5413 & \underline{0.6928} & \textbf{0.7385} \\
    H3K4me1 & \underline{0.8942} & 0.8183 & \textbf{0.9465} \\
    H3K36me3 & \underline{0.8404} & 0.5417 & \textbf{0.8905} \\
    H3K9me3 & 0.1271 & \underline{0.5883} & \textbf{0.5911} \\
    H3K27me3 & \underline{0.9429} & 0.9197 & \textbf{0.9925} \\\hline
    \end{tabular}
    \captionsetup{labelfont=bf}
    \renewcommand{\baselinestretch}{1.0}
    \caption[Comparison of different inputs of baseline with AUPRC]{This table represents performance with AUPRC. Best predictive performance of each histone modification is shown in bold.}
    \label{t7}
\end{table}

\begin{table}[H]%加入table環境指令以控制表格的位置、編號與標題,[h]代表將表格置於here,其他位置的標示請參考手冊
    \centering
    \begin{tabular}{lccc}
    \hline
    Model & $Inception_{Meth}$ & $Inception_{DNA}$ & $Inception_{DNA+Meth}$ \\\hline
    H3K4me3 & 0.9204 & \underline{0.9806} & \textbf{0.9927} \\
    H3K27ac & 0.8719 & \underline{0.93} & \textbf{0.9549} \\
    H3K4me1 & \underline{0.9481} & 0.9144 & \textbf{0.9787} \\
    H3K36me3 & \underline{0.9789} & 0.9119 & \textbf{0.9886} \\
    H3K9me3 & 0.8198 & \underline{0.9648} & \textbf{0.9762} \\
    H3K27me3 & \underline{0.9705} & 0.9532 & \textbf{0.9969} \\\hline
    \end{tabular}
    \captionsetup{labelfont=bf}
    \renewcommand{\baselinestretch}{1.0}
    \caption[Comparison of different inputs of inception with AUROC]{This table represents performance with AUROC. Best predictive performance of each histone modification is shown in bold.}
    \label{t8}
\end{table}

\begin{table}[H]%加入table環境指令以控制表格的位置、編號與標題,[h]代表將表格置於here,其他位置的標示請參考手冊
    \centering
    \begin{tabular}{lccc}
    \hline
    Model & $Inception_{Meth}$ & $Inception_{DNA}$ & $Inception_{DNA+Meth}$ \\\hline
    H3K4me3 & 0.6971 & \underline{0.8775} & \textbf{0.9474} \\
    H3K27ac & 0.5741 & \underline{0.7134} & \textbf{0.7934} \\
    H3K4me1 & \underline{0.9033} & 0.8514 & \textbf{0.9588} \\
    H3K36me3 & \underline{0.852} & 0.6141 & \textbf{0.9208} \\
    H3K9me3 & 0.1585 & \underline{0.6936} & \textbf{0.8114} \\
    H3K27me3 & \underline{0.96} & 0.9507 & \textbf{0.9965} \\\hline
    \end{tabular}
    \captionsetup{labelfont=bf}
    \renewcommand{\baselinestretch}{1.0}
    \caption[Comparison of different inputs of inception with AUPRC]{This table represents performance with AUPRC. Best predictive performance of each histone modification is shown in bold.}
    \label{t9}
\end{table}

It is obvious that models, which integrating DNA sequences and methylation signal, get better performance than models, which only extract feature from DNA sequences. We use underline to represent the better performance in same framework, that comparing the models trained by only DNA sequences and only DNA methylation signal. Interestingly, the models predicting H3K4me1, H3K36me3 and H3K27me3 only input the features from DNA methylation signal has betters results than from DNA sequences. Therefore, DNA methylation provide more insight for model and improve prediction.

\subsection{Cross cell lines} \label{cross}
According to a series of experiments, we confirm that introducing techniques and features are good for our model. Next, we train $Inception_{DNA+Meth}$ on  different cell lines, including A549, GM12878 and K562, to get more information across cell lines. Similarly, we select nine partitions of each cell line to train, and evaluate on one partition of each cell line. Number of data points for training and testing separately are 4760181 and 528909. We show the mean of performance in the Table\ref{t10} and Table\ref{t11}.

\begin{table}[H]%加入table環境指令以控制表格的位置、編號與標題,[h]代表將表格置於here,其他位置的標示請參考手冊
    \centering
    \begin{tabular}{lccc}
    \hline
    Model & $Inception_{Meth}$ & $Inception_{DNA}$ & $Inception_{DNA+Meth}$ \\\hline
    H3K4me3 & 0.9214 & \underline{0.9657} & \textbf{0.984} \\
    H3K27ac & 0.8422 & \underline{0.8659} & \textbf{0.9288} \\
    H3K4me1 & \underline{0.8772} & 0.8511 & \textbf{0.9441} \\
    H3K36me3 & \underline{0.9576} & 0.8998 & \textbf{0.9774} \\
    H3K9me3 & 0.8018 & \underline{0.9518} & \textbf{0.9719} \\
    H3K27me3 & \underline{0.9557} & 0.909 & \textbf{0.9915} \\\hline
    \end{tabular}
    \captionsetup{labelfont=bf}
    \renewcommand{\baselinestretch}{1.0}
    \caption[Comparison of different inputs across cell lines with AUROC]{This table represents performance with AUROC. Best predictive performance of each histone modification is shown in bold.}
    \label{t10}
\end{table}

\begin{table}[H]%加入table環境指令以控制表格的位置、編號與標題,[h]代表將表格置於here,其他位置的標示請參考手冊
    \centering
    \begin{tabular}{lccc}
    \hline
    Model & $Inception_{Meth}$ & $Inception_{DNA}$ & $Inception_{DNA+Meth}$ \\\hline
    H3K4me3 & 0.6863 & \underline{0.83} & \textbf{0.9006} \\
    H3K27ac & 0.6231 & \underline{0.6662} & \textbf{0.7862} \\
    H3K4me1 & \underline{0.7223} & 0.6694 & \textbf{0.8565} \\
    H3K36me3 & \underline{0.8803} & 0.7859 & \textbf{0.9415} \\
    H3K9me3 & 0.2386 & \underline{0.6865} & \textbf{0.7987} \\
    H3K27me3 & \underline{0.8995} & 0.8497 & \textbf{0.9834} \\\hline
    \end{tabular}
    \captionsetup{labelfont=bf}
    \renewcommand{\baselinestretch}{1.0}
    \caption[Comparison of different inputs across cell lines with AUPRC]{This table represents performance with AUPRC. Best predictive performance of each histone modification is shown in bold.}
    \label{t11}
\end{table}

\section{Discussion} \label{discuss}
Furthermore, we want to examine whether our model definitely learn insight of crossing cell lines, by visualizing representation of each data point, called feature vector, that is input of fully connected layers in our model. First, we use the method of dimension reduction, t-distributed stochastic neighbor embedding (t-SNE), to covert each feature vector to two dimensions and three dimensions. Next, we utilize scatter plot to observe distribution of converted vector in each cell line through plotting different colors to annotate which cell line is it.

In order to observe which input feature provide the most information of crossing cell lines, we individually visualize the feature vectors generated by models, which models are trained according to previous experiment in Section\ref{cross}.

\begin{figure}[H]
    \centering
    \includegraphics[width=1\columnwidth]{body/figure/figure17.png}
    \captionsetup{labelfont=bf}
    \renewcommand{\baselinestretch}{1.0}
    \caption[Visualization of feature vectors generated by $Inception_{DNA}$]{Visualization of feature vectors generated by $Inception_{DNA}$.}
    \label{f17}
\end{figure}

\begin{figure}[H]
    \centering
    \includegraphics[width=1\columnwidth]{body/figure/figure18.png}
    \captionsetup{labelfont=bf}
    \renewcommand{\baselinestretch}{1.0}
    \caption[Visualization of feature vectors generated by $Inception_{Meth}$]{Visualization of feature vectors generated by $Inception_{Meth}$.}
    \label{f18}
\end{figure}

\begin{figure}[H]
    \centering
    \includegraphics[width=1\columnwidth]{body/figure/figure19.png}
    \captionsetup{labelfont=bf}
    \renewcommand{\baselinestretch}{1.0}
    \caption[Visualization of feature vectors generated by $Inception_{DNA+Meth}$]{Visualization of feature vectors generated by $Inception_{DNA+Meth}$.}
    \label{f19}
\end{figure}

By above visualization, we clearly observe variations of distribution. As long as adding input feature of DNA methylation, the model will learn cell line-specific information via DNA methylation, and automatically discriminate different biosamples within whole feature vectors. This phenomenon is not observed on model only using input features of DNA sequences.